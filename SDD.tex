
\documentclass{article}

\usepackage{pgffor}
\usepackage{tikz}
\usetikzlibrary{fit,positioning}

\title{SpeedCola SDD}
\date{2025-09-01}
\author{..}

\begin{document}
\maketitle
\newpage
\pagenumbering{arabic}

\section{Introduction}
  \begin{enumerate}
    \item  \textbf{Descripcion General}

  El proposito de este documento es describir tanto la arquitectura como la implementacion de 
  SpeedCola en aras de facilitar su uso, modificacion, prubeas y resolucion de errores.

  \item \textbf{Software's Scope}

  SpeedCola es una aplicacion web que facilita la conexion entre proveedores de servicio y consumidores.

  \end{enumerate}

  \subsection{In-Scope Funcitonalities}
    \begin{enumerate}
      \item \textbf{Registro y Autenticacion}
        \begin{enumerate}
          \item Los usuarios pueden crear cuentas
          \item Los usuarios pueden validar sus cuentas mediante documentos
          \item Las cuentas son autenticadas mediante correo electronico.
        \end{enumerate}
      \item \textbf{Manejo de Perfiles}
        \begin{enumerate}
          \item Los consumidores pueden crear y editar cuentas
          \item Los proveedores pueden crear  y editar cuentas
          \item Los proveedores pueden 
        \end{enumerate}
      Permite la edición de perfiles tanto de consumidores como de 
      proveedores, incluyendo información personal, experiencia y tarifas.  
      \item \textbf{Oferta de Servicios}
        Permite la creacion de 
      \item \textbf{Busqueda de Servicios}
      \item \textbf{Registro y Autenticacion}
    \end{enumerate}

      Gestiona la publicación, búsqueda y visualización de servicios disponibles. 

      Módulo de comunicación  

      Permite a los usuarios a interactuar y crear contratos mediante chats con proveedores. 

      Módulo de calificaciones y reseñas   

      Contiene la funcionalidad para la evaluación mutua entre consumidores y proveedores al finalizar un servicio.  

      Módulo de notificaciones  

      Crea y gestiona las notificaciones con un sistema basado en eventos tales como: 

      Ofertas nuevas 

      Confirmaciones de pago 

      Reseñas recibidas  

      Mensajes 
      
\section{Dise\~no General}
  
  \subsection{Arquitectura del Sistema}
    \begin{tikzpicture}

    \begin{scope}[shift={(1,1)},local bounding box=A]
    \draw (0,0) rectangle (11.3,1.5);
    \draw (0.2,0.2) rectangle (3.2,1);
    \draw (3.7,0.2) rectangle (6.7,1);
    \draw (7.2,0.2) rectangle (10.2,1);
    \node[scale=0.8] at (1.65,0.5) {$S_{x_{1}}$};
    \node[scale=0.8] at (5.5,0.5) {$S_{x_{2}}$};
    \node[scale=0.8] at (8.5,0.5) {$S_{x_{3}}$};
    \node[scale=0.8] at (10.65,0.5) {$S_{x}$};
    \end{scope}

    \draw[->] (0,0) -- (A.west);
    \end{tikzpicture}
  //todo diagrama de aws
  //todo diagrama de apis
  \subsection{Justificacion de la INfraestructura}

\section{Consideraciones Tecnicas}
\newpage

\section{Arquitectura}
\newpage


\section{Detalles del Dise\~no}
  \foreach \module in {autentificacion, chat, browser}
  {
    \subsection{Modulo: internals \module}
    parse some yaml files with the methods etc
    \newline
    \begin{tabular}{ l l l l l  }
      \hline
      Metodo & Proposito & entrada & Salida & Status \\
      \hline
          GET & 2 & 3 & 3 & 200\\
          GET & 2 & 3 & 3 & 200\\ 
          GET & 2 & 3 & 3 & 200\\
        
      \hline
    \end{tabular}
  }
\end{document}
