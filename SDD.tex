\documentclass{article}

\title{SpeedCola SDD}
\date{2025-09-01}
\author{..}

\begin{document}
\maketitle
\newpage
\pagenumbering{arabic}

\section{Introduction}
\begin{enumerate}
  \item  \textbf{Descripcion General}

El proposito de este documento es describir tanto la arquitectura como la implementacion de 
SpeedCola en aras de facilitar su uso, modificacion, prubeas y resolucion de errores.

\item \textbf{Software's Scope}

SpeedCola es una aplicacion web que facilita la conexion entre proveedores de servicio y consumidores.

\end{enumerate}

\subsection{In-Scope Funcitonalities}
  \begin{enumerate}
    \item \textbf{Registro y Autenticacion}
      \begin{enumerate}
        \item Los usuarios pueden crear cuentas
        \item Los usuarios pueden validar sus cuentas mediante documentos
        \item Las cuentas son autenticadas mediante correo electronico.
      \end{enumerate}
      Encargado el registro, inicio de sesión y validación de identidad de usuarios mediante documentos oficiales. 
    \item \textbf{Manejo de Perfiles}
    Permite la edición de perfiles tanto de consumidores como de proveedores, incluyendo información personal, experiencia y tarifas.  
    \item \textbf{Oferta de Servicios}
      Permite la creacion de 
    \item \textbf{Busqueda de Servicios}
    \item \textbf{Registro y Autenticacion}
  \end{enumerate}

    Gestiona la publicación, búsqueda y visualización de servicios disponibles. 

    Módulo de comunicación  

    Permite a los usuarios a interactuar y crear contratos mediante chats con proveedores. 

    Módulo de calificaciones y reseñas   

    Contiene la funcionalidad para la evaluación mutua entre consumidores y proveedores al finalizar un servicio.  

    Módulo de notificaciones  

    Crea y gestiona las notificaciones con un sistema basado en eventos tales como: 

    Ofertas nuevas 

    Confirmaciones de pago 

    Reseñas recibidas  

    Mensajes 
\end{document}
