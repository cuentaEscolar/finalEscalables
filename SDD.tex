
\documentclass{article}
\usepackage[a4paper,margin=1in,footskip=0.25in]{geometry}
\usepackage{svg}
\usepackage{listings}
\usepackage{pgffor}
\usepackage{csvsimple}
\usepackage{tikz}
\usepackage{ragged2e}
\usetikzlibrary{fit,positioning}
\usetikzlibrary{shapes.multipart}
\usetikzlibrary{positioning}
\usetikzlibrary{shadows}
\usetikzlibrary{calc}
\usepackage{array}
\renewcommand{\arraystretch}{1.1}
\usepackage{tikz}
\usetikzlibrary{shapes.multipart}
\usetikzlibrary{positioning}
\usetikzlibrary{shadows}
\usetikzlibrary{calc}
\usepackage{multicol}
\usepackage{pdflscape}

\newenvironment{localsize}[1]
{%
  \clearpage
  \let\orignewcommand\newcommand
  \let\newcommand\renewcommand
  \makeatletter
  \input{bk#1.clo}%
  \makeatother
  \let\newcommand\orignewcommand
}
{%
  \clearpage
}

\makeatletter
\title{SpeedCola Software Design Document}
\date{\today}% Date of issue and status
\author{SpeeCola inc}% Issuing organization

\begin{document}

\maketitle
IEEE Std 1016™-2009

\newpage
\pagenumbering{arabic}

  \iffalse  
  ⎯ Date of issue and status
  ⎯ Scope
  ⎯ Issuing organization
  ⎯ Authorship (responsibility or copyright information)
  ⎯ References
  \fi
  \section{Identificaci\'on del SDD}
  \paragraph {Fecha de Redaccion} Redactado el \today \space por SpeedCola S.A de C.V.\newline

  SpeedCola es una aplicacion web que facilita la conexion entre proveedores de servicio y consumidores. 
  %context y Summary
  Este documento estipula la arquitectura, el dise\~no y 
    las decisiones sobre que se compremetera y su porque. Su proposito es describir tanto la arquitectura como la implementacion del sitio web de  
  SpeedCola en aras de facilitar su uso, modificacion, pruebas y resolucion de errores.

    En este mismo se incluyen las siguientes vistas del sistema:
    \begin{multicols}{3}
    \begin{enumerate}

      \item Contexto
      \item Composici\'on 
      \item Logica
      \item De Interacci\'on
      \item Recursos

    \end{enumerate}
  \end{multicols}

  \subsection{Alcance del proyecto}

  \begin {multicols}{2}
    \begin{enumerate}
      \item \textbf{Registro y Autenticacion}
        \begin{enumerate}
          \item Los usuarios pueden crear cuentas
          \item Los usuarios pueden validar sus cuentas mediante documentos
          \item Las cuentas son autenticadas mediante correo electronico.
        \end{enumerate}
      \item \textbf{Manejo de Perfiles}
        \begin{enumerate}
          \item Los consumidores pueden crear y editar cuentas
          \item Los proveedores pueden crear  y editar cuentas
          \item Los proveedores pueden editar sus horarios y sus tarifas
        \end{enumerate}
      \item \textbf{Oferta de Servicios}
        \begin{enumerate}
          \item Los proveedores pueden crear servicios
        \end{enumerate}
      \item \textbf{Busqueda de Servicios}
        \begin{enumerate}
          \item Los consumidores pueden buscar y filtrar servicios
        \end{enumerate}
      \item \textbf{Registro y Autenticacion}
        \begin{enumerate}
          \item Los usuarios pueden autenticarse con documentos oficiales
        \end{enumerate}
      \item \textbf{Módulo de comunicación } 
        \begin{enumerate}
          \item Permite a los usuarios a interactuar y crear contratos mediante chats con proveedores. 
        \end{enumerate}
      \item \textbf{Módulo de calificaciones y reseñas   }
        \begin{enumerate} 
          \item Permite a los usuarios pueden evaluar los servicios
          \item Permite a los oferentes de servicios evaluar a los usuarios
          \item Mantiene restricciones sobre las situaciones en las que se puede evaluar un servicio
        \end{enumerate} 
      \item \textbf{Módulo de notificaciones}
      \begin{enumerate}
        \item Crea y gestiona las notificaciones con un sistema basado en eventos tales como: 
          \begin{enumerate}
            \item Anuncio de servicios
            \item Cancelacion de servicios
            \item Confirmacion de servicios
            \item Ofertas nuevas 
            \item Confirmaciones de pago 
            \item Reseñas recibidas  
            \item Mensajes 
          \end{enumerate}
      \end{enumerate} 

    \end{enumerate}
  \end{multicols}
  Los siguientes terminos seran utilizados a traves del documento.

  \paragraph{Glosario}
  \begin{enumerate}
    \item AWS: Amazon Web Services
    \item VPC: Virtual Private Connection
  \end{enumerate}

  \section{Stakeholders del dise\~no}
    Los posibles interesados son los due\~nos del producto, los miembros del equipo de desarrollo y el director del proyecto. Al 
    due\~no del proyecto le interesa que el sitio web sea atractivo, que sea replicable en multiples zonas geograficas , sea 
    resiliente a cambios intempestivos en la carga de trabajo, y que minimize su costo. Por el otro lado, al equipo de desarrollo le interesa tener Interfaces consistentes para poder trabajar en un producto fconsistente. 
    Finalmente, al administrador del proyecto le interesa un dise\~no modular para poder paralelizar el trabajo. En este documento, se respondera a todas estas necesidades.
  \section{Vistas de dise\~no}

  \begin{enumerate}
      \item Composici\'on 
      \item Logica
      \item De Interacci\'on
      \item Recursos
  \end{enumerate}


\subsection{Contexto}
  El siguiente punto de vista concierne el punto de vista de los dos tipos de usuarios.

  \begin{figure}[!htb]
    \begin{center}
      \includesvg[width=400pt]{Use Case.svg}
    \end{center}
    \caption{Los usuarios se muestran con figuras de palos, y muestran los "servicios" esperados que cada respectivo tipo de cliente pudiese utilizar.}\label{fig:}
  \end{figure}
  

\section{Vista L\'ogica}
\subsection{uml}

\begin{figure}[!htb]
  \begin{center}
    \includesvg[width = 400pt]{./UML SpeedCola.drawio.svg}
  \end{center}
  \caption{}\label{fig:}
\end{figure}


\section{Dise\~no General}
  
\section{Arquitectura}
  \subsection{Arquitectura del Sistema}
  \begin{figure}[!htb]
    \begin{center}
      \includesvg[width=500pt]{BasicArch.svg}
    \end{center}
    \caption{Arquitectura Basica}\label{fig: }
  Dise\~no basico de la Arquitectura.
  \end{figure}

  \begin{figure}[!htb]
    \begin{center}
      \includesvg[width=550pt]{archDiagram.svg}
    \end{center}
    \caption{Arquitectura en la nube.}\label{fig: Implementacion en la Nube}
    El Administrador es una entidad que tiene permisos especiales para inspeccionar el funcionamiento de la Infraestructura.
    Los Usuarios regulares no tienen semejantes privilegios.
    Tanto el Administrador como los Usuarios Regulares interactuan a traves del Internet Gateway.
    El Network Load Balancer recibe el trafico del Internet Gateway y lo redirige a la tabla de routeo privada. La direccion a la que la redirija depende de la VPC en la que se esta.
    La tabla de routeo privada 

  \end{figure}
  \subsection{temp fix}
  \paragraph{ Entidades }
      \iffalse{
    NOTE 1—Design attributes can be thought of as questions about design elements. The answers to those questions are
    the values of the attributes. All the questions can be answered, but the content of the answer will depend upon the
    nature of the entity. The collection of answers provides a complete description of an entity. Attribute descriptions
    should include references and design considerations such as tradeoffs and assumptions when appropriate. In some
    cases, attribute descriptions may have the value none.
    NOTE 2—Design attributes have been generalized from the concept of design entity attribute (which appeared in
    IEEE Std 1016-1998 and applied only to design entities) to apply to design entities, design relationships, and design
    constraints.
    NOTE 3—Use of the design attributes in 4.6.2.1 through 4.6.2.3 ensures compatibility with IEEE Std 1016-1998. Other
    design attributes required as a part of specific design viewpoints are defined with those viewpoints in Clause 5. Some
    design attributes [such as subordinates (see 5.3.2.2)] can be more usefully represented as design relationships. This was
    endif
      } \fi 

  \foreach \module in { cloudformation}
  {
    \csvautotabular {./designElements/\module.csv}
  }

  \subsection{Justificacion de la Infraestructura}
  La Arquitectura fue diens\~ada para facilitar la separaci\'on de necesidades, escalabilidad y seguridad.
    El modulo de entrada y salida es practicamente el buscador, y solo se asegura de gg

\section{Consideraciones Tecnicas}
\newpage

\newpage


\section{Detalles del Dise\~no}
  \foreach \module in {
    auth, contract, chat, browser, search, events, notifs}
  {
    \subsection{Modulo: \module}
    \csvautotabular {./api/\module.csv}
  }
\newpage
\section{Dise\~no de los Datos}
  //poner las tablas

  \begin{landscape}
      \begin{center}
      \end{center}
  \end{landscape}
\section{Dise\~no de Interfaces}


\end{document}
