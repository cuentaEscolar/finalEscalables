
\documentclass{article}
\usepackage[a4paper,margin=1in,footskip=0.25in]{geometry}
\usepackage{svg}
\usepackage{listings}
\usepackage{pgffor}
\usepackage{csvsimple}
\usepackage{tikz}
\usetikzlibrary{fit,positioning}
\usetikzlibrary{shapes.multipart}
\usetikzlibrary{positioning}
\usetikzlibrary{shadows}
\usetikzlibrary{calc}
\usepackage{array}
\renewcommand{\arraystretch}{1.1}
\usepackage{tikz}
\usetikzlibrary{shapes.multipart}
\usetikzlibrary{positioning}
\usetikzlibrary{shadows}
\usetikzlibrary{calc}
\usepackage{multicol}
\usepackage{pdflscape}

\newenvironment{localsize}[1]
{%
  \clearpage
  \let\orignewcommand\newcommand
  \let\newcommand\renewcommand
  \makeatletter
  \input{bk#1.clo}%
  \makeatother
  \let\newcommand\orignewcommand
}
{%
  \clearpage
}

\makeatletter
\title{SpeedCola Software Design Document}
\date{\today}% Date of issue and status
\author{SpeeCola inc}% Issuing organization

\begin{document}

\maketitle
IEEE Std 1016™-2009

\newpage
\pagenumbering{arabic}

\section{Introducci\'on}
  \subsection{Identificaci\'on}
  Redactado el \today.
  El siguiente documento fue redactado for el equipo de dise\~no SpeedCola el \today . 

  SpeedCola es una aplicacion web que facilita la conexion entre proveedores de servicio y consumidores.

  estipula la arquitectura, el dise\~no y 
    las decisiones sobre que se compremetera y su porque. Este documento es de interes para los accionistas de SpeedCola, asi como a todos los equipos de desarrollo. 
    En este mismo se incluyen las siguientes vistas del sistema:
    \begin{multicols}{2}
    \begin{enumerate}

      \item Contexto
      \item Composici\'on 
      \item Logica
      \item De Interacci\'on
      \item Recursos

    \end{enumerate}
  \end{multicols}

  \begin{enumerate}
    \item  \textbf{Descripcion General}

  El proposito de este documento es describir tanto la arquitectura como la implementacion de 
  SpeedCola en aras de facilitar su uso, modificacion, pruebas y resolucion de errores.



  \end{enumerate}

  \subsection{In-Scope Funcitonalities}
  \begin {multicols}{2}
    \begin{enumerate}
      \item \textbf{Registro y Autenticacion}
        \begin{enumerate}
          \item Los usuarios pueden crear cuentas
          \item Los usuarios pueden validar sus cuentas mediante documentos
          \item Las cuentas son autenticadas mediante correo electronico.
        \end{enumerate}
      \item \textbf{Manejo de Perfiles}
        \begin{enumerate}
          \item Los consumidores pueden crear y editar cuentas
          \item Los proveedores pueden crear  y editar cuentas
          \item Los proveedores pueden editar sus horarios y sus tarifas
        \end{enumerate}
      \item \textbf{Oferta de Servicios}
        \begin{enumerate}
          \item Los proveedores pueden crear servicios
        \end{enumerate}
      \item \textbf{Busqueda de Servicios}
        \begin{enumerate}
          \item Los consumidores pueden buscar y filtrar servicios
        \end{enumerate}
      \item \textbf{Registro y Autenticacion}
        \begin{enumerate}
          \item Los usuarios pueden autenticarse con documentos oficiales
        \end{enumerate}
      \item \textbf{Módulo de comunicación } 
        \begin{enumerate}
          \item Permite a los usuarios a interactuar y crear contratos mediante chats con proveedores. 
        \end{enumerate}
      \item \textbf{Módulo de calificaciones y reseñas   }
        \begin{enumerate} 
          \item Permite a los usuarios pueden evaluar los servicios
          \item Permite a los oferentes de servicios evaluar a los usuarios
          \item Mantiene restricciones sobre las situaciones en las que se puede evaluar un servicio
        \end{enumerate} 
      \item \textbf{Módulo de notificaciones}
      \begin{enumerate}
        \item Crea y gestiona las notificaciones con un sistema basado en eventos tales como: 
          \begin{enumerate}
            \item Anuncio de servicios
            \item Cancelacion de servicios
            \item Confirmacion de servicios
            \item Ofertas nuevas 
            \item Confirmaciones de pago 
            \item Reseñas recibidas  
            \item Mensajes 
          \end{enumerate}
      \end{enumerate} 

    \end{enumerate}
  \end{multicols}





      \begin{enumerate}
        \item [x] vista logica
        \item vista de procesos
        \item vista de despliegue
            (logico)
            (modulos)
            (cloudformation etc.)
        \item vista fisica
          despliegue fisico
        \item vista de escenarios

      \end{enumerate}
      


\section{Contexto}
awawawa

\subsection{Punto de vista de Uso}

  \begin{figure}[!htb]
    \begin{center}
      \includesvg[width=400pt]{Use Case.svg}
    \end{center}
    \caption{Ejemplo de Uso}\label{fig:}
  \end{figure}
  

\section{Vista L\'ogica}
\subsection{uml}

\begin{figure}[!htb]
  \begin{center}
    \includesvg[width = 400pt]{./UML SpeedCola.drawio.svg}
  \end{center}
  \caption{}\label{fig:}
\end{figure}


\section{Dise\~no General}
  
  \subsection{Arquitectura del Sistema}
  \begin{figure}
    \begin{center}
      \includesvg[width=500pt]{BasicArch.svg}
    \end{center}
    \caption{}\label{fig: Arquitectura Basica}
  Dise\~no basico de la Arquitectura.
  \end{figure}

  \begin{figure}
    \begin{center}
      \includesvg[width=550pt]{archDiagram.svg}
    \end{center}
    \caption{}\label{fig: Implementacion en la Nube}
    Los usuarios regulares interactuan con el sistema a traves del Internet Gateway.
    Los administradores ti

  \end{figure}

  \subsection{Justificacion de la Infraestructura}
  La Arquitectura fue diens\~ada para facilitar la separaci\'on de necesidades, escalabilidad y seguridad.
    El modulo de entrada y salida es practicamente el buscador, y solo se asegura de gg

\section{Consideraciones Tecnicas}
\newpage

\section{Arquitectura}
\newpage


\section{Detalles del Dise\~no}
  \foreach \module in {
    auth, contract, chat, browser, search, events, notifs}
  {
    \subsection{Modulo: \module}
    \csvautotabular {./api/\module.csv}
  }
\newpage
\section{Dise\~no de los Datos}
  //poner las tablas

  \begin{landscape}
      \begin{center}
      \end{center}
  \end{landscape}
\section{Dise\~no de Interfaces}


\end{document}
